\documentclass[10pt,oneside,a4paper]{article}
\usepackage[utf8]{inputenc}
\usepackage{amsmath}
\usepackage{indentfirst}
\usepackage{enumitem}
\usepackage[spanish]{babel}
\usepackage[export]{adjustbox}
\usepackage{graphicx}
\graphicspath{ {img/} }
\usepackage{listings}
\usepackage{subfig}
\usepackage{cite}
\usepackage{pgfplots}

\addtolength{\oddsidemargin}{-.300in}
\addtolength{\evensidemargin}{-.300in}
\addtolength{\textwidth}{0.600in}
\addtolength{\topmargin}{-.300in}
\addtolength{\textheight}{0.600in} %1.75

\begin{document}
\begin{titlepage}

\title{\Huge Procesadores Gráficos y Aplicaciones en Tiempo Real  \\[0.7in] \LARGE Sistema de partículas por GPU\\[3.6in]}
\date{}
\author{Álvaro Muñoz Fernández\\
Iván Velasco González}
\maketitle
\thispagestyle{empty}
\end{titlepage}

\section{Introducción}
El objetivo de esta práctica consiste en la implementación de un modelo de partículas básico implementado en shaders de cómputo que se ejecutarán en GPU. Además se proponen tareas adicionales, de las que se ha optado por la implementación de la ordenación de partículas (también realizada en shaders de cómputo) para una correcta visualización de las partículas.\\

En caso de que la aplicación no se ejecute de forma correcta por falta de las \textit{dll} sera neceario modificar la variable de entorno de Visual Studio en Propiedades del proyecto \rightarrow Depuración y poner en entorno PATH=\$(ProjectDir)..\textbackslash bin;\%PATH\%; en caso de ejecutar una version de 32 bits y  PATH=\$(ProjectDir)..\textbackslash bin\textbackslash x64;
\%PATH\%

\section{Aplicación cliente}
La aplicación se ha implementado en C++ partiendo de un proyecto vacío en el que se han añadido las librerías necesarias para el desarrollo. Se ha utilizado \textit{freeGLUT} para la inicialización del contexto, \textit{GLEW} para la inicialización de extensiones, \textit{GLM} para las funciones matemáticas y estructuras de datos algebraicas, y \textit{FreeImage} para la carga de texturas, aunque finalmente no se ha cargado ninguna textura en el proyecto.\\

El proyecto se divide en funciones de inicialización, funciones de cómputo y funciones de renderizado que son llamadas en los puntos de ejecución donde son necesarias. Además se emplean estructuras de datos para almacenar la información correspondiente a los programas utilizados.\\

Adicionalmente se ha añadido una comprobación para que la ejecución termine si se detecta que el tamaño de WorkGroup supera el límite devuelto por el driver.

\section{Sistema de partículas}
El sistema de partículas que se ha implementado es el propuesto por Mike Bailey, que incluye solamente colisiones entre las partículas y una esfera definida mediante su posición y radio. Opcionalmente podía extenderse este sistema, pero nosotros hemos optado por la implementación de ordenación y renderizado con transparencia.\\

\begin{figure}[h!tbp]
\centering
\includegraphics[width=.8\linewidth]{img/particles.png}
\caption{Instantánea del sistema de partículas}
\end{figure}

El sistema de partículas se almacena en tres buffers que contienen las posiciones, velocidades y colores de las partículas.

\section{Ordenación}
\subsection{Inicialización}
Antes de proceder a la ordenación de las partículas es necesario inicializar dos buffers extra. Uno de ellos almacena los índices que se emplearán en la llamada \textit{DrawElements()} para realizar el dibujado ordenado, y el otro almacenará las distancias de cada partícula a la cámara.\\

Este paso previo se realiza en una shader de cómputo que exclusivamente almacena, de forma paralela, la distancia de cada una de las partículas en el array de distancias.

\subsection{Ordenación}
La ordenación se realiza en una nueva shader de cómputo que realiza varias pasadas. Hemos optado por una implementación de \textit{bitonic sort}, que recibe el buffer de distancias y el buffer de índices y realiza la ordenación solamente del buffer de índices.\\

Con este buffer ordenado puede realizarse el renderizado ordenado a través de un VAO sin necesidad de modificar los buffers con información de partículas. Se ha tomado esta decisión porque aunque incrementa el uso de memoria, simplifica el número de accesos ya que se escribe sobre un solo buffer de valores enteros utilizando la información leída de un buffer auxiliar, en lugar de modificar los 3 buffers de información de cada partícula, cada uno de ellos con 3 elementos float por cada partícula.

\section{Renderizado}
\subsection{Billboards}
\begin{figure}[h!tbp]
\centering
\includegraphics[width=.8\linewidth]{img/billboards.png}
\caption{Varios billboards orientados a cámara}
\end{figure}
Para dibujar cada una de las partículas se ha creado una shader de geometría que recibe las posiciones de las partículas como puntos y las traslada a posiciones en el espacio de cámara. En este espacio el eje Y está orientado perpendicularmente a la cámara, por lo que al generarse un quad siempre estará orientado hacia la cámara. De esta forma se consigue el efecto billboard con facilidad, generándose los quads sobre los que dibujar la partícula de forma eficiente en la shader de geometría. Además se generan coordenadas UV en cada vértice de la partícula que permitirán colorear correctamente la partícula en la shader de fragmentos.

\subsection{Texturizado}

\begin{figure}[h!tbp]
\centering
\includegraphics[width=.8\linewidth]{img/proceduraltexture.png}
\caption{Visualización de una partícula con transparencia}
\end{figure}
Para colorear las partículas se utiliza el color que se asigna aleatoriamente al inicio de la simulación. En lugar de utilizar una textura como máscara, se ha generado procedimentalmente la máscara de las partículas. Mediante la distancia al centro de la partícula se calcula el porcentaje de opacidad de la partícula, decayendo este cuanto más lejos del centro está un fragmento. Así se consigue una forma esférica suavizada para las partículas, con una transparencia variable que permite que algunas zonas sean translúcidas y se vea con facilidad el resultado de la aplicación del blending.

\section{Funcionamiento de la demo}
\subsection{Tamaño de partícula}
Se ha añadido funcionalidad para alterar el tamaño de partícula en tiempo de ejecución. Esto puede hacerse mediante las teclas \textbf{+} y \textbf{-}.

\subsection{Pausa de simulación}
La simulación se puede pausar y reanudar presionando la tecla \textbf{P}.

\subsection{Reinicio de simulación}
La simulación se puede reiniciar presionando la tecla \textbf{R}.

\subsection{Alternar ordenado}
El ordenado de partículas puede activarse y desactivarse presionando la tecla \textbf{T}.

\subsection{Estadísticas}
Presionando la tecla \textbf{L} se muestran por consola las estadísticas de la media de frametime y framerate de los últimos 50 frames.

\subsection{Cámara}
La cámara se puede manipular mediante las teclas \textbf{W}, \textbf{A}, \textbf{S}, \textbf{D} y con el movimiento del ratón mientras se mantiene presionado el botón izquierdo. Además se puede alterar la elevación de la cámara utilizando la tecla \textbf{Espacio} para ascender y \textbf{Z} para descender.

\section{Análisis de resultados}
\subsection{Metodología}
Para las pruebas se han realizado varias ejecuciones en las que se ha alterado el parámetro del tamaño de WorkGroup de las shaders, se han tomado 10 medidas para ejecuciones con cada uno de los parámetros y se han calculado las medias para cada ejecución. Las medidas se han desglosado entre el tiempo dedicado al cálculo de físicas, el tiempo de ordenamiento de las partículas y el tiempo de dibujado de las partículas.\\

Las pruebas se han realizado con un número de partículas de 1024x1024.\\

Las especificaciones del equipo de pruebas son las siguientes:
\begin{itemize}
  \item Procesador: AMD Ryzen 1600X 3.6GHz
  \item Memoria RAM: 16GB DDR4 3000Mhz
  \item Tarjeta gráfica: Nvidia GTX 1060 6GB
\end{itemize}

\subsection{Media de resultados por ejecución}\begin{figure}[h!tbp]
\centering
\includegraphics[width=.8\linewidth]{img/hist.png}
\caption{Histograma de las medias de tiempos}
\end{figure}

\subsection{Tablas de resultados}

\begin{tabular}{ |p{1.3cm}||p{2cm}|p{2cm}|p{2cm}| }
 \hline
 \multicolumn{4}{|c|}{Frametimes para tamaño de WorkGroup 16} \\
 \hline
 Iteración& Físicas&Ordenación&Rendering\\
 \hline
1   & 0.35158&117.294&4.47528\\
2&0.50754&117.188&4.38544\\
3&0.56222&117.225&4.27188\\
4&0.55212&117.019&4.16744\\
5&0.56998&117.021&3.96658\\
6&0.5778&116.986&3.76458\\
7&0.62348&116.977&3.73252\\
8&0.66744&116.97&3.70164\\
9&0.70276&116.96&3.60998\\
10&0.71646&116.71&3.43508\\
Media&0.583138&117.035&3.951042\\
 \hline
\end{tabular}\\\vspace{1cm}

\begin{tabular}{ |p{1.3cm}||p{2cm}|p{2cm}|p{2cm}| }
 \hline
 \multicolumn{4}{|c|}{Frametimes para tamaño de WorkGroup 128} \\
 \hline
 Iteración& Físicas&Ordenación&Rendering\\
 \hline
1&0.49302&110.307&4.32266\\
2&0.54266&110.307&4.27152\\
3&0.55578&110.156&4.02068\\
4&0.5653&110.002&3.75822\\
5&0.5773&109.968&3.69862\\
6&0.58566&109.99&3.57418\\
7&0.62238&110.002&3.38232\\
8&0.64676&109.803&3.18426\\
9&0.67342&109.816&3.07414\\
10&0.70036&109.83&2.971\\
Media&0.596264&110.0181&3.62576\\
 \hline
\end{tabular}\\\vspace{1cm}

\begin{tabular}{ |p{1.3cm}||p{2cm}|p{2cm}|p{2cm}| }
 \hline
 \multicolumn{4}{|c|}{Frametimes para tamaño de WorkGroup 512} \\
 \hline
 Iteración& Físicas&Ordenación&Rendering\\
 \hline
1&0.56121&109.93&4.07228\\
2&0.54918&109.948&4.15572\\
3&0.55176&109.77&3.81522\\
4&0.55892&109.702&3.72278\\
5&0.60886&109.705&3.59522\\
6&0.64248&109.755&3.3695\\
7&0.6717&109.786&3.2267\\
8&0.67832&109.58&3.03026\\
9&0.67306&109.57&2.9145\\
10&0.57638&109.986&4.27416\\
Media&0.612295556&109.7557778&3.567117778\\
 \hline
\end{tabular}

\subsection{Conclusiones}
De los datos obtenidos hemos concluído que el grueso del tiempo de ejecución se debe a la ordenación de partículas. Nuestra implementación de bitonic no es la más óptima, haciendo accesos a memoria global constantes, por lo que el tiempo de ejecución de la ordenación corresponde a más de un 90\% del tiempo total de cada frame.\\

Además se puede observar que al variar el tamaño del WorkGroup hay ligeras variaciones en el tiempo de ejecución, afectando principalmente a la etapa de ordenación. Cuando el tamaño se incrementa se reducen los tiempos de ordenación, aunque el tiempo de físicas parece no cambiar y los resultados entran dentro del margen de error. En el caso de la etapa de dibujado no hay efecto, ya que el tamaño de grupo no interviene en el dibujado.

\end{document}